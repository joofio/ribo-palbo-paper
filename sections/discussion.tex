The aim of this study was to evaluate the real-world use of palbociclib and ribociclib in combination with ET for HR+/HER2$-$ and comparing this drug class with traditional endocrine therapy. Few real-world evidence studies of palbociclib and ribociclib used in daily clinical practice have been published identifying clinical benefit, patient profile and sequencing of treatment, with even less evidence of use of palbociclib in Portugal.

When comparing with clinical trials, regarding patient profile, in our study, 51\% had visceral metastasis and 35\% bone only disease comparing with 49\% and 38\% in PALOMA-2, and 60\% and 25\% in PALOMA-3, respectively \cite{rugoImpactPalbociclibLetrozole2018,cristofanilliFulvestrantPalbociclibFulvestrant2016a}.
As for ribociclib, MONALEESA-7 \cite{tripathyRibociclibEndocrineTherapy2018} has 24\% and MONALEESA-2 has 40\% \cite{hortobagyiUpdatedResultsMONALEESA22018} and our study has 30\%.



Of note, the range of median PFS for first-line palbociclib was 15.5–25.5 months, which is shorter than 27.6 months observed in a post hoc analysis of the PALOMA-2 clinical trial with extended follow-up \cite{rugoImpactPalbociclibLetrozole2018}, but in line with RWE studies (13.3–20.2 months) \cite{harbeckCDK4InhibitorsHR2021}.
As for ribociclib, median survival time was not reached wether in OS and PFS. So we can at least say that the median PFS is longer than 50 months. This is longer that the median progression-free survival of 23.8 months (95\% CI 19.2–not reached) reported in the MONALEESA-7 trial \cite{tripathyRibociclibEndocrineTherapy2018} and longer than  25.3 months (95\% CI 23.0–30.3) in the MONALEESA-2 trial \cite{hortobagyiUpdatedResultsMONALEESA22018}. However, the HT group has a median PFS of 13.6 months, which is in tune with the reported values in the literature.

Regarding the comparison between HT and CDK4/6i first line, we found out that neither OS and PFS have significant changes when compared HT and Palbociclib 1st line. This is an unexpected result, since we would expect that the addition of palbociclib would increase at least the PFS significantly.
However, the difference is significant for Ribociclib. We also made a cox regression, adjusted for drug (inside HT) which was not significant with p values over 0.2.

When comparing with propensity scores weighting, we found out that ribociclib is significantly better than palbociclib both in terms of OS and PFS. Our findings suggest that ribociclib could be the optimal approach for treating HR+, HE- metastic breast cancer, providing a median OS of over 40 months and median PFS of around 42 months. 


%possivelmente os de ET continua a ser melhor opçcao e so quem desenvolve resistencia passa para CDK4/6i
