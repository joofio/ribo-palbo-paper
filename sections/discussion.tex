The aim of this prospective study was to evaluate the real-world use of palbociclib and ribociclib in combination with ET for HR+/HER2$-$ and comparing this drug class with traditional hormonotherapy. Few real-world evidence studies of palbociclib and ribociclib used in daily clinical practice have been published identifying clinical benefit, patient profile and sequencing of treatment, with even less evidence of use of palbociclib in Portugal.


%In RENATA study, we included 60\% of patients treated with palbociclib in first line and 40\% in second or more lines, and palbociclib was combined with letrozole in 62\% of patients and in 36\% with fulvestrant. As the PALOMA-3 trial (21\%) we treated 20\% premenopausal women with palbociclib, all of them underwent to OS/OA. In the prospective POLARIS study, palbociclib was used in first line in 70\% of patients, in combination with letrozole/anastrazole, fulvestrant and exemestane in 57\%, 39\% and 4\%, respectively [5]. Regarding patient profile, in our study, 29\% had visceral metastasis and 27\% bone only disease comparing with 49\% and 38\% in PALOMA-2, and 60\% and 25\% in PALOMA-3, respectively [3, 4]. Important to mention, 55\% of patients were naïve for advanced treatment, and 20\% were ‘de novo’ stage 4, lower than 37\% of ‘de novo’ stage 4 in PALOMA-2 and 30\% in PALOMA-3 study. The ORR in RENATA study was 45\% in first line (with 3.6\% of CR), and 25\% in second line (without CR), similar rates than PALOMA-2 (ORR: 42\%) and PALOMA-3 (ORR:19\%, without CR). The overall CB was 82\%, 86\% in first line (85\% in PALOMA-2) and 38\% in second line (67\% in PALOMA-3). Regarding performance status, 65\% of patients reported symptomatic improvement since starting with palbociclib. The median PFS in first line setting was 36.7 months longer than median PFS reported in PALOMA-2 study (24.8 months) [3].