This study aimed to evaluate the real-world use of palbociclib and ribociclib in combination with ET for HR+/HER2$-$ and compare this drug class with traditional endocrine therapy. Few real-world evidence studies of palbociclib and ribociclib used in daily clinical practice have been published identifying clinical benefits, patient profiles, and sequencing of treatment, with even less evidence for the Portuguese population.

When comparing with clinical trials, regarding patient profile, in our study, 51\% had visceral metastasis and 35\% had bone-only metastases compared with 49\% and 38\% in PALOMA-2, and 60\% and 25\% in PALOMA-3, respectively \cite{rugoImpactPalbociclibLetrozole2018,cristofanilliFulvestrantPalbociclibFulvestrant2016a}.
As for ribociclib and bone-only metastases, MONALEESA-7 \cite{tripathyRibociclibEndocrineTherapy2018} has 24\% and MONALEESA-2 has 40\% \cite{hortobagyiUpdatedResultsMONALEESA22018} and our study has 30\%.
Regarding menopausal status, our study has 20\% premenopausal and 80\% postmenopausal. The trials were conducted with postmenopausal women only.


Of note, the range of median PFS for first-line palbociclib was 15.5–25.5 months, which is shorter than 27.6 months observed in a post hoc analysis of the PALOMA-2 clinical trial with extended follow-up \cite{rugoImpactPalbociclibLetrozole2018}, but in line with RWE studies (13.3–20.2 months) \cite{harbeckCDK4InhibitorsHR2021}. When assessed with only letrozol as a combination, the median PFS increased to 28.6 months [95\% CI 25.5-not reached].
Additionally, analyzing the postmenopausal women subgroup, palbociclib showed a median PFS of 16.3 months [95\% CI 12.9 -20]. Furthering analysis of the postmenopausal and with letrozol, the median was 47.6 months [95\% 25.6-2–not reached].

As for ribociclib, median survival time was not reached whether in OS and PFS. So we can at least say that the median PFS is longer than 50 months. This is longer than the median progression-free survival of 23.8 months (95\% CI 19.2–not reached) reported in the MONALEESA-7 trial \cite{tripathyRibociclibEndocrineTherapy2018} and longer than  25.3 months (95\% CI 23.0–30.3) in the MONALEESA-2 trial \cite{hortobagyiUpdatedResultsMONALEESA22018}. 
Regarding the subgroup analysis of postmenopausal women and postmenopausal women treated ribociclib in combination with letrozol, the median was not reached.

When directly comparing ribociclib and palbociclib without any adjustments, one might deduce that ribociclib is superior to palbociclib. However, after adjusting for confounding variables, there is no significant difference between the two inhibitors in terms of Progression-Free Survival (PFS) or Overall Survival (OS) as indicated in \ref*{tab:cox}. This observation is further corroborated by the lower plots in \ref*{fig:interest}, where even a subgroup analysis of CDK4/6i combined solely with letrozole reveals a trend towards nonsignificance.

In the first-line comparison between Endocrine Therapy (ET) and CDK4/6 inhibitors (CDK4/6i), there is no significant impact on Overall Survival (OS) by either treatment, whether the CDK4/6i is combined with both agents or solely with letrozole. With respect to Progression-Free Survival (PFS), ribociclib demonstrates superior efficacy in both combination therapies (HR=0.29) as well as when paired only with letrozole (HR=0.28). Additionally, palbociclib exhibits significant improvement in PFS when combined with letrozole (HR=0.50).


When comparing with propensity scores weighting, we found out that ribociclib is significantly better than palbociclib for PFS and OS, providing a median OS of over 40 months and median PFS of around 42 months. Adjusted for the weighted variables, Ribociclib is no longer significantly better for OS and has a p-value of exactly 0.05 for PFS. This is in line with the results of the Cox regression analysis.


