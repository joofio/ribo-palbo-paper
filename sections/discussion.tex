The aim of this prospective study was to evaluate the real-world use of palbociclib and ribociclib in combination with ET for HR+/HER2$-$ and comparing this drug class with traditional hormonotherapy. Few real-world evidence studies of palbociclib and ribociclib used in daily clinical practice have been published identifying clinical benefit, patient profile and sequencing of treatment, with even less evidence of use of palbociclib in Portugal.

When comparing with clinical trials, regarding patient profile, in our study, 51\% had visceral metastasis and 35\% bone only disease comparing with 49\% and 38\% in PALOMA-2, and 60\% and 25\% in PALOMA-3, respectively \cite{rugoImpactPalbociclibLetrozole2018,cristofanilliFulvestrantPalbociclibFulvestrant2016a}.
As for ribociclib, MONALEESA-7 \cite{tripathyRibociclibEndocrineTherapy2018} has 24\% and MONALEESA-2 has 40\% \cite{hortobagyiUpdatedResultsMONALEESA22018} and our study has 30\%.



Of note, the range of median PFS for first-line palbociclib was plus letrozole in RWE studies was 15.5–25.5 months, which is shorter than 27.6 months observed in a post hoc analysis of the PALOMA-2 clinical trial with extended follow-up \cite{rugoImpactPalbociclibLetrozole2018}, but in line with RWE studies (13.3–20.2 months) \cite{harbeckCDK4InhibitorsHR2021}.
As for ribo


\subsection{propensity Scores}

The importance of quality data in observational studies cannot be overstated. Unlike randomized controlled trials (RCTs), where randomization helps to balance both observed and unobserved covariates between the treatment and control groups, observational studies are often fraught with selection biases, confounding variables, and imbalances in baseline characteristics. Researchers typically have no control over the assignment of subjects to treatment or control groups, leading to potential biases that can significantly skew results. Well-structured and rich datasets can provide a wealth of information that allows for more accurate control of these confounding factors. By including a variety of variables that might influence the outcome, data richness enables the use of statistical techniques like matching, stratification, or weighting to create comparable treatment and control groups, thereby mimicking the conditions of an RCT to some extent.

One of the critical ways to partially mitigate the issues inherent in observational studies is through the use of propensity score methods, such as Average Treatment Effect (ATE) and Average Treatment effect on the Treated (ATT) weighted Kaplan-Meier curves. These methods seek to balance the distribution of observed covariates between treatment and control groups, thereby reducing selection bias. Once balanced, the survival curves can more accurately reflect the true impact of the treatment, providing results that are closer to what might be observed in a randomized study. In essence, propensity score methods help to level the playing field by reweighting or resampling the original data based on the probability of receiving treatment, allowing for a more fair comparison between the treatment and control groups.

That being said, it's crucial to remember that even the most sophisticated statistical techniques can only control for observed confounders; hidden biases due to unmeasured or unknown variables can still persist. Additionally, the quality of the propensity score model is heavily reliant on the data available, underlining the need for comprehensive data collection and thorough exploratory data analysis. The application of methods like ATE and ATT weighted Kaplan-Meier curves is not a substitute for good data but a complement to it. In sum, while quality data and sophisticated statistical methods can't fully replicate the conditions of a randomized trial, they can substantially improve the validity and reliability of findings from observational data.


%In RENATA study, we included 60\% of patients treated with palbociclib in first line and 40\% in second or more lines, and palbociclib was combined with letrozole in 62\% of patients and in 36\% with fulvestrant. As the PALOMA-3 trial (21\%) we treated 20\% premenopausal women with palbociclib, all of them underwent to OS/OA. In the prospective POLARIS study, palbociclib was used in first line in 70\% of patients, in combination with letrozole/anastrazole, fulvestrant and exemestane in 57\%, 39\% and 4\%, respectively [5]. Regarding patient profile, in our study, 29\% had visceral metastasis and 27\% bone only disease comparing with 49\% and 38\% in PALOMA-2, and 60\% and 25\% in PALOMA-3, respectively [3, 4]. Important to mention, 55\% of patients were naïve for advanced treatment, and 20\% were ‘de novo’ stage 4, lower than 37\% of ‘de novo’ stage 4 in PALOMA-2 and 30\% in PALOMA-3 study. The ORR in RENATA study was 45\% in first line (with 3.6\% of CR), and 25\% in second line (without CR), similar rates than PALOMA-2 (ORR: 42\%) and PALOMA-3 (ORR:19\%, without CR). The overall CB was 82\%, 86\% in first line (85\% in PALOMA-2) and 38\% in second line (67\% in PALOMA-3). Regarding performance status, 65\% of patients reported symptomatic improvement since starting with palbociclib. The median PFS in first line setting was 36.7 months longer than median PFS reported in PALOMA-2 study (24.8 months) [3].

