

\subsection{Study Design}

%This prospective study was designed in 2015 when palbociclib was approved by the National Administration of Medicines—ANMAT (Administración Nacional de Medicamentos, Alimentos y Tecnología Médica). The aim of the study was to evaluate the clinical benefit, side effects and long-term survival of patients with HR+/HER2− ABC treated with palbociclib plus ET in different lines of treatment between October 2015 and August 2019. Inclusion criteria: pre and postmenopausal women, men, Oestrogen Receptor positive (defined by ER expression ≥1% of tumour cell by immunohistochemistry, IHC) and HER2 negative (by IHC and/or amplification assay) in primary tumour or metastatic site after biopsy, that have received at least one cycle of palbociclib in advanced setting. 

This retrospective study was designed in 2022. The aim of the study was to evaluate the clinical benefit, side effects and long-term survival of patients with HR+/HER2$-$ that started treatment with CDK4\/6 inhibitors plus hormonotherapy in different lines of treatment between the 14th of March 2017 and the 31st of December 2021. The follow-up period was set until June 2022. 
Inclusion criteria: pre and postmenopausal women, men, Oestrogen Receptor positive \% $($defined by ER expression $\geq$ 1 \% of tumour cell by immunohistochemistry, IHC)
and HER2 negative $($by IHC and/or amplification assay) in the primary tumour or metastatic site after biopsy.
Exclusion criteria: Patients that had only ambulatory medication, and patients involved in clinical trials.
The comparison group was defined by a population of patients, that were treated with hormone therapy as first-line between 2015 and 13 of match 2017.

The evaluation of effectiveness will involve overall survival and progression-free analysis. We will compare the three different cyclin-dependent kinase inhibitors in terms of efficacy in real-world patients and will also compare the effectiveness of this class of drug against traditional hormonotherapy. We will also compare them with the clinical trials when possible. 


%All data were collected from original medical records from baseline to last visit or death. Data included: demographic information, age at first diagnosis and age at the beginning of treatment with palbociclib, clinical characteristics and performance status by Eastern Cooperative Oncology Group scale (ECOG). Treatment-related data: loco-regional and neo/adjuvant systemic treatment, number and type of treatments in advance setting before palbociclib, type of treatment beyond palbociclib progression, treatment strategy in premenopausal women (ovarian suppression / ovarian ablation, OS/OA), palliative radiation therapy before or during palbociclib treatment and partner of palbociclib in different lines. Metastatic data at the beginning of palbociclib: ‘de novo’ metastatic disease, site of metastases (bone, soft tissue, visceral, visceral and bone, central nervous system-CNS with or without other site), and metastatic site at palbociclib progression. Patients predisposition: side effects by frequency and grade (NCI-CTCAE version 4.0), starting dose and number of patients with dose interruption, delay, reduction or treatment discontinuation

\subsection{Data collection}
All data were collected from original medical records from baseline to last visit or death.
The data was collected from  Instituto Português de Oncologia – Porto (IPO-P).  table \ref{tab:stats_ipop_cdk} shows a comparison between the groups.
%%%copied
Data included: demographic information, age at first diagnosis and age at the beginning of treatment with palbociclib, clinical characteristics and performance status by Eastern Cooperative Oncology Group scale (ECOG). Treatment-related data: loco-regional and neo/adjuvant systemic treatment, number and type of treatments in advance setting before palbociclib, type of treatment beyond palbociclib progression, treatment strategy in premenopausal women (ovarian suppression / ovarian ablation, OS/OA), palliative radiation therapy before or during palbociclib treatment and partner of palbociclib in different lines. Metastatic data at the beginning of palbociclib: ‘de novo’ metastatic disease, site of metastases (bone, soft tissue, visceral, visceral and bone, central nervous system-CNS with or without another site), and metastatic site at palbociclib progression. Patients predisposition: side effects by frequency and grade (NCI-CTCAE version 4.0), starting dose and number of patients with dose interruption, delay, reduction or treatment discontinuation

 
\begin{table}
\caption{Descriptive statistics of cyclin-dependent kinase inhibitors group}
\label{tab:stats_ipop_cdk}
\begin{tabular}[t]{lllll}
\toprule
  & Abemaciclib & Palbociclib & Ribociclib & Overall\\
\midrule
 & (N=12) & (N=247) & (N=106) & (N=365)\\
\addlinespace[0.3em]
\multicolumn{5}{l}{\textbf{Age at treatment start}}\\
\hspace{1em}Mean (SD) & 58.8 (11.5) & 59.2 (11.7) & 58.2 (10.7) & 58.9 (11.4)\\
\hspace{1em}Median [Min, Max] & 58.5 [39.0, 74.0] & 60.0 [28.0, 84.0] & 58.0 [32.0, 79.0] & 59.0 [28.0, 84.0]\\
\addlinespace[0.3em]
\multicolumn{5}{l}{\textbf{Treatment Line}}\\
\hspace{1em}1st Line & 3 (25.0\%) & 127 (51.4\%) & 98 (92.5\%) & 228 (62.5\%)\\
\hspace{1em}2nd+ Lines & 9 (75.0\%) & 120 (48.6\%) & 8 (7.5\%) & 137 (37.5\%)\\
\addlinespace[0.3em]
\multicolumn{5}{l}{\textbf{PFS}}\\
\hspace{1em}Censored & 9 (75.0\%) & 84 (34.0\%) & 76 (71.7\%) & 169 (46.3\%)\\
\hspace{1em}Dead & 3 (25.0\%) & 163 (66.0\%) & 30 (28.3\%) & 196 (53.7\%)\\
\addlinespace[0.3em]
\multicolumn{5}{l}{\textbf{OS}}\\
\hspace{1em}Censored & 10 (83.3\%) & 148 (59.9\%) & 88 (83.0\%) & 246 (67.4\%)\\
\hspace{1em}Dead & 2 (16.7\%) & 99 (40.1\%) & 18 (17.0\%) & 119 (32.6\%)\\
\addlinespace[0.3em]
\multicolumn{5}{l}{\textbf{Stage}}\\
\hspace{1em}I & 1 (8.3\%) & 22 (8.9\%) & 7 (6.6\%) & 30 (8.2\%)\\
\hspace{1em}II & 4 (33.3\%) & 75 (30.4\%) & 22 (20.8\%) & 101 (27.7\%)\\
\hspace{1em}III & 3 (25.0\%) & 75 (30.4\%) & 18 (17.0\%) & 96 (26.3\%)\\
\hspace{1em}IV & 2 (16.7\%) & 65 (26.3\%) & 46 (43.4\%) & 113 (31.0\%)\\
\hspace{1em}Missing & 2 (16.7\%) & 10 (4.0\%) & 13 (12.3\%) & 25 (6.8\%)\\
\bottomrule
\end{tabular}
\end{table}


\begin{table}[]
\caption{Descriptive statistics of palbociclib and ribociclib group vs hormonotherapy}
\label{tab:stats_ipop_control}
\begin{tabular}[t]{llll}
\toprule
  & CDK4/6 & Chemo & Overall\\
\midrule
 & (N=225) & (N=43) & (N=268)\\
\addlinespace[0.3em]
\multicolumn{4}{l}{\textbf{Age at treatment start}}\\
\hspace{1em}Mean (SD) & 59.1 (11.5) & 60.1 (12.4) & 59.3 (11.6)\\
\hspace{1em}Median [Min, Max] & 59.0 [28.0, 84.0] & 62.0 [34.0, 85.0] & 60.0 [28.0, 85.0]\\
\addlinespace[0.3em]
\multicolumn{4}{l}{\textbf{PFS}}\\
\hspace{1em}Censored & 123 (54.7\%) & 2 (4.7\%) & 125 (46.6\%)\\
\hspace{1em}Dead & 102 (45.3\%) & 41 (95.3\%) & 143 (53.4\%)\\
\addlinespace[0.3em]
\multicolumn{4}{l}{\textbf{OS}}\\
\hspace{1em}Censored & 168 (74.7\%) & 8 (18.6\%) & 176 (65.7\%)\\
\hspace{1em}Dead & 57 (25.3\%) & 35 (81.4\%) & 92 (34.3\%)\\
\addlinespace[0.3em]
\multicolumn{4}{l}{\textbf{Estrogen Receptor}}\\
\hspace{1em}+ & 225 (100\%) & 42 (97.7\%) & 267 (99.6\%)\\
\hspace{1em}- & 0 (0\%) & 1 (2.3\%) & 1 (0.4\%)\\
\addlinespace[0.3em]
\multicolumn{4}{l}{\textbf{Progesterone Receptor}}\\
\hspace{1em}+ & 168 (74.7\%) & 27 (62.8\%) & 195 (72.8\%)\\
\hspace{1em}- & 57 (25.3\%) & 16 (37.2\%) & 73 (27.2\%)\\
\addlinespace[0.3em]
\multicolumn{4}{l}{\textbf{Stage}}\\
\hspace{1em}I & 16 (7.1\%) & 3 (7.0\%) & 19 (7.1\%)\\
\hspace{1em}II & 55 (24.4\%) & 20 (46.5\%) & 75 (28.0\%)\\
\hspace{1em}III & 62 (27.6\%) & 11 (25.6\%) & 73 (27.2\%)\\
\hspace{1em}IV & 75 (33.3\%) & 2 (4.7\%) & 77 (28.7\%)\\
\hspace{1em}Missing & 17 (7.6\%) & 7 (16.3\%) & 24 (9.0\%)\\
\bottomrule
\end{tabular}
\end{table}

\subsection{Statistical Analysis}
%copied
R was used for statistical analysis. Demographic, clinical characteristics and side effects were analysed using descriptive statistics (count, percentages and median/range). Kaplan–Meier test was used to determine the median PFS and OS in the entire population and subgroups. Log-rank test was used for comparisons of PFS and OS among different subgroups. Cox Regression was used to assess feature importance and impact. All statistical tests were two-sided, and the significance level was 0.05. 




%Pretende-se que os dados venham do Instituto Português de Oncologia – Porto (IPO-P). Pretende-se utilizar a base de dados do hospital dos últimos 5 anos.
%O estudo será registado e respeitará todos os requisitos éticos de aprovação a comissão de ética de cada instituição participante. Caso a instituição tenha um encarregado de Proteção de Dados (EPD), este será contactado a fim de dar seu parecer, e caso necessário, a sua opinião para melhorar possíveis pontos relacionados à segurança de dados.

%The goal is to identify the clinical or biological variables that have the most impact on a specific outcome. To achieve this, considering the statistical models found, we can model things in terms of time series or survival trees to group outcomes and the clusters found. With this, we can analyze the effectiveness of the clinical course, probabilities of recurrence, or survival rates by subgroup. These models and relationships can form the basis of a clinical decision support system and can be crucial for making better healthcare decisions.

%Some of the techniques used are unsupervised techniques such as k-means, DBSCAN, or hierarchical clustering.
%The goal is to identify the clinical or biological variables that have the most impact on a specific outcome. To achieve this, considering the statistical models found, we can model things in terms of time series or survival trees to group outcomes and the clusters found. With this, we can analyze the effectiveness of the clinical course, probabilities of recurrence, or survival rates by subgroup. These models and relationships can form the basis of a clinical decision support system and can be crucial for making better healthcare decisions.

%Some of the techniques used are unsupervised techniques such as k-means, DBSCAN, or hierarchical clustering.
