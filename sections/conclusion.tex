In conclusion, our findings underscore the efficacy of CDK4/6 inhibitors in real-world settings. While we cannot definitively assert that palbociclib surpasses endocrine therapy in terms of Overall Survival—a facet not extensively explored in prominent clinical trials—we can confidently affirm the impact of CDK4/6i on Progression-Free Survival. This assertion aligns with clinical trial outcomes and real-world data further substantiates these findings.
Delving deeper into the characteristics of the patient population, including safety profiles, economic implications, and quality of life metrics, would be insightful. Additionally, a thorough examination of biomarkers within the population could offer invaluable insights. We intend to explore these facets in subsequent publications.
It’s imperative to note that our data is sourced from a singular institution, limiting the capability of generalization of our results to a broader population. Nonetheless, we posit that this study lays a foundational groundwork for future research in this domain. While our evidence is rooted in observational data, and we’ve made adjustments for known confounders, the potential for residual confounding remains. Although the use of propensity score matching enhances the comparative robustness between the groups, the presence of unmeasured confounders cannot be entirely ruled out. Furthermore, the small sample size of our study limits the statistical power of our findings. We hope that our study will serve as a springboard for future research in this domain, and we look forward to furthering our research in this area.