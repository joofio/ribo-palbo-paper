Currently, metastatic breast cancer is difficult to treat. Patients with hormone receptor $($HR$)$-positive and HER2-negative, the most common subtype, typically undergo endocrine therapy. Therefore, new treatments can be very useful in improving quality of life, reducing toxicity, and decreasing scenarios of hormonal resistance.
Medications from the group of cyclin-dependent kinase inhibitors appear as a potential improvement in the therapeutic approach to advanced breast cancer. Within this group, there are palbociclib, ribociclib, and abemaciclib. Cyclin-dependent kinases 4 and 6 $($CDK4/6$)$ are responsible for regulating the cell cycle at the transition between the G1 and S phases. In many neoplasms, this cycle is deregulated, and it promotes uncontrolled cell proliferation. It is then possible for these medications to have better effectiveness. These medications were approved by INFARMED, I.P. after an analysis of the therapeutic value they offer. For this purpose, data from clinical trials conducted with these medications were essentially used. The MONALEESA \cite{hortobagyiUpdatedResultsMONALEESA22018, slamonPhaseIIIRandomized2018, tripathyRibociclibEndocrineTherapy2018} studies were used for ribociclib, PALOMA \cite{vermaPalbociclibCombinationFulvestrant2016, rugoImpactPalbociclibLetrozole2018, finnCyclindependentKinaseInhibitor2015a} for palbociclib, and MONARCH \cite{goetzMONARCHAbemaciclibInitial2017, sledgeMONARCHAbemaciclibCombination2017} for abemaciclib.
These studies focused on testing the hypothesis of treating CDK4/6 inhibitors in combination with an aromatase inhibitor or fulvestrant as an alternative to the gold standard. In these studies, it was concluded that they brought a significant increase in effectiveness, justifying their use in clinical practice.
However, this evaluation was based on clinical trials with very specific inclusion and exclusion criteria and in a highly controlled environment. It is then vital to study how these new molecules compare to current practice in terms of treatment effectiveness in a real-world setting.